
After acquiring the models from a set of training grasps, we present the robot with a test (query) object. The goal is to find a generalisation of the training grasp that is likely according to all of the model types simultaneously. First of all, we obtain a point cloud for the test object, and thus an object model. We then combine every contact model with that object model, so as to obtain a set of {\em query densities}, one for each link with a contact model defined for the example grasp. The $i$-th query density $\qd_i$ is a density modelling where the $i$-th link can be placed, in the equilibrium state, with respect to the surface of a new object. 

From the query densities, a candidate equilibrium grasp state is generated as follows. We randomly pick a link $i$. We randomly sample, from the corresponding query density $\qd_i$, a pose for link $i$. We sample, from the configuration model $\hc$, a final hand configuration that is compatible with the pose selected for link $i$, and then we compute from forward kinematics the poses of all the remaining hand links. This defines a candidate equilibrium grasp state. This is then augmented with a reach to grasp trajectory that will lead to it. The reach to grasp trajectory is selected that finishes closest to the candidate equilibrium grasp state. Finally we refine the equilibrium grasp and reach to grasp by performing a simulated annealing search in the space of equilibrium state wrist poses and hand configurations, so as to maximise the grasp likelihood. Grasp likelihood is the product of the hand configuration density, the reach to grasp density, and the query densities for all the hand links. We repeat the entire process a number of times. The optimisation procedure generates many possible grasps, each with its likelihood. The following subsections explain how to estimate query densities, and how grasp optimisation is carried out.

\subsection{Query Density Computation}

A query density \eqref{eq:qd} is expressed, for a hand link and an object model, as a density for the pose of that hand link relative to the object. Intuitively the query density encourages a finger link to make contact with the object at locations that have similar local surface curvature to that in the training example. Specifically, we use $K_{Q_i}$ kernels centred on the set of weighted finger link poses:
\begin{equation}
\qd_i(s) \simeq \sum^{K_{Q_i}}_{j=1} w_{ij} \mathcal{N}_3(p|{\hat{p}_{ij}}, \sigma_{p}) \Theta(q|{\hat{q}_{ij}}, \sigma_{q})%, \quad i = 1, ..., N_L
\label{eq:qd.approx}
\end{equation}
with $j$-th kernel centre $({\hat{p}_{ij}}, {\hat{q}_{ij}}) = \hat{s}_{ij}$, and where all weights were normalised $\sum_j w_{ij} = 1$. When a test object is presented, a set of query densities $\mathcal{Q}^g$ is calculated for the equilibrium state for each training grasp $g$. The set $\mathcal{Q}^g =\{\qd_i^g\}$ has $N^g_Q=N^g_M$ members, one for each contact model $M_i^g$ in $\mathcal{M}^g$.