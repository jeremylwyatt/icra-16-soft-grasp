Transferring dexterous grasps to novel objects is a challenging problem. One approach is to machine learn solutions with techniques able to perform powerful generalisation. Another is to use an underactuated hand to cope with shape variation. In this paper we combine the benefits of both approaches by learning grasps for underactuated hands. Underactuated hands exploit the contacts that occur during grasping to achieve a wide variety of final grasp configurations. The final grasp configuration depends not only on the final hand pose, but also on the object shape, and on the reach to grasp trajectory. An interesting challenge is to use machine learning to exploit these interactions. The key technical challenge in applying machine learning to grasping with underactuated hands is to learn the right trajectory for a particular object shape so as to achieve a good grasp of a particular type. 

One approach would be to learn the typical contact interactions that occur during a grasp, and to generate new grasps that reproduce these. The contact interactions are, however, rather complex and variable, even given small variations in object shape and friction. Therefore we tackle the problem by implicitly encoding the contact interactions in terms of the approach trajectory. Our method learns both the desired final contacts, the final hand shape, and the hand pose during the reach to grasp trajectory. We build on our previous work on one-shot learning of grasps that transfer to novel objects, employing a product of experts formulation. The method is able to learn a flexible grasp from examples generated in physics simulation. It copes with partial and noisy shape information for the test objects. The method requires no knowledge of the human defined object category, either when learning or performing transfer.

% SUBSECTION BETTER?
\subsection{RELATED WORK}
Previous work in learning generalisable grasps falls broadly into two classes. One class of approaches utilises the shape of common object parts or their appearance to generalise grasps across object categories \cite{saxena2008b,detry2013a,herzog2014a, kroemer2012a}. This works well for low DoF hands. Another class of approaches captures the global properties of the hand shape either at the point of grasping, or during the approach \cite{ben2012generalization}. This global hand shape can additionally be associated with global object shape, allowing generalisation by warping grasps to match warps of global object shape \cite{hillenbrand2012transferring}. This second class works well for high DoF hands, but generalisation is more limited. We have previously achieved the advantages of both classes, generalising grasps across object categories with high DoF hands. In this paper we go beyond this, learning and generalising grasps for under-actuated hands.