\documentclass[letter, 10pt, conference]{ieeeconf}

\IEEEoverridecommandlockouts

\overrideIEEEmargins

\usepackage{amssymb,url,amsmath}
\usepackage{algorithm}
\usepackage{graphicx}
\usepackage{epstopdf}
\usepackage{textcomp}
\usepackage{color}
\usepackage{ifpdf}
\usepackage{url}
\usepackage{enumerate}
\usepackage{array}
\usepackage{flushend}
\usepackage{multirow,booktabs}
\usepackage{hyperref}
\usepackage{subfig}
\usepackage{mdwlist}
\usepackage[utf8]{inputenc}
\usepackage{tcolorbox}
\newcommand{\alg}{Algorithm~}
\newtcolorbox[auto counter]{algorithmbox}[2][]{colback=red!5!white,colframe=red!75!black,fonttitle=\bfseries, title=\alg\thetcbcounter: #2,#1}

\newcommand{\eq}{Eq.~}
\newcommand{\fig}{Fig.~}
\newcommand{\tab}{Tab.~}
\newcommand{\sect}{Sec.~}

\newcommand{\re}{R}
\newcommand{\rf}{R}
\newcommand{\rl}{L}
\newcommand{\om}{O}
\newcommand{\cm}{M}
\newcommand{\fm}{F}
\newcommand{\hc}{C}
\newcommand{\rg}{G}
\newcommand{\qd}{Q}
\newcommand{\gpm}{T}
\newcommand{\coll}{W}
\newcommand{\pdf}{\mathbf{pdf}}

\newcommand{\argmax}[1]{\underset{#1}{\operatorname{argmax}}\medspace}

\graphicspath{{images/}}

\title{\LARGE \bf
Learning and Inference of Dexterous Grasps for Novel Objects with Underactuated Hands
}

\author{Marek Kopicki$^{1}$ and Carlos J. Rosales$^{2}$ and Hamal Marino$^{2}$ and Marco Gabiccini$^{2}$ and Jeremy L. Wyatt$^{1}$% <-this % stops a space
\thanks{*This work was supported by EC-FP7-ICT-600918, PacMan.}% <-this % stops a space
\thanks{$^{1}$Marek Kopicki and Jeremy L. Wyatt are with the Intelligent Robotics Laboratory, School of Computer Science,
        University of Birmingham, Birmingham, B15 2TT, UK.}
        %{\tt\small \{msk,jlw\}@cs.bham.ac.uk}}%
\thanks{$^{2}$Carlos J. Rosales, Hamal Marino and Marco Gabiccini are with Centro Piaggio, Universita di Pisa, Pisa, Italy.}
               %{\tt\small carlos.rosales@for.unipi.it, hamal.marino@centropiaggio.unipi.it, m.gabiccini@ing.unipi.it}}%
\thanks{Correspond to: {\tt\small M.S.Kopicki at cs.bham.ac.uk}}%
}
\begin{document}


\maketitle
\thispagestyle{empty}
\pagestyle{empty}


%%%%%%%%%%%%%%%%%%%%%%%%%%%%%%%%%%%%%%%%%%%%%%%%%%%%%%%%%%%%%%%%%%%%%%%%%%%%%%%%
\begin{abstract}
Recent advances have been made in learning of grasps for fully actuated hands. A typical approach learns the target locations of finger links on the object. When a new object must be grasped, new finger locations are generated, and a collision free reach-to-grasp trajectory is planned. Such a division of labour fails to transfer directly to underactuated hands, which improve grasp reliability via contacts with the object during grasping. In this paper we present a method for learning transferrable grasps for underactuated hands. Our approach learns not only the desired final grasp, and also good grasping trajectories, from a rigid body simulation. This enables us to learn how to approach the object and close the underactuated hand from a variety of poses. Our method does not rely on explicit representation of the contact sequence. The core learning method uses product of experts. This allows grasp transfer to novel objects and works despite partial shape reconstruction of less than 25\% of the surface. From nine training grasps on three objects the method transferred grasps to previously unseen, novel objects, that differ significantly from the training objects, with an 80\% success rate. We move beyond previous work by: i) showing the ability to learn transferrable grasps for underactuated hands; ii) extending our learning method to work with multiple training examples for each grasp type; iii) extending our method to work with multiple reach to grasp trajectories.
\end{abstract}


%%%%%%%%%%%%%%%%%%%%%%%%%%%%%%%%%%%%%%%%%%%%%%%%%%%%%%%%%%%%%%%%%%%%%%%%%%%%%%%%
\section{INTRODUCTION}
\label{sec:introduction}
Transferring dexterous grasps to novel objects is a challenging problem. One approach is to machine learn solutions with techniques able to perform powerful generalisation. Another is to use an underactuated hand to cope with shape variation. In this paper we combine the benefits of both approaches by learning grasps for underactuated hands. Underactuated hands exploit the contacts that occur during grasping to achieve a wide variety of final grasp configurations. The final grasp configuration depends not only on the final hand pose, but also on the object shape, and on the reach to grasp trajectory. An interesting challenge is to use machine learning to exploit these interactions. The key technical challenge in applying machine learning to grasping with underactuated hands is to learn the right trajectory for a particular object shape so as to achieve a good grasp of a particular type. 

\begin{figure}
  \centering
  \begin{tabular}{ccc}
  \includegraphics[width=0.45\linewidth]{mug_1_small.jpg} &
  \includegraphics[width=0.45\linewidth]{mug_2_small.jpg} \\
  \end{tabular}
 \caption{{We want transferrable grasps that are robust to different initial hand-object poses, and thus different interactions during reach to grasp, thus reaching similar final grasp states. We achieve this by learning a set of trajectories that, associated with a model of the final grasp state, form an attractor basin around that state.}}
  \label{fig:two_grasps}
\end{figure}

One approach would be to learn the typical contact interactions that occur during a grasp, and to generate new grasps that reproduce these. The contact interactions are, however, rather complex and variable, even given small variations in object shape and friction. Therefore we tackle the problem by implicitly encoding the contact interactions in terms of the approach trajectory. Our method learns both the desired final contacts, the final hand shape, and possible sequences of hand pose during the reach to grasp trajectory. We build on our previous work on one-shot learning of grasps that transfer to novel objects, employing a product of experts. The novel technical contribution is that in this paper we show how to learn not only final grasp, but also the approach trajectory and control strategy for closing the dexterous hand. In particular, we learn a bundle of multiple trajectories that will all likely lead to a similar final stable grasp. We enable this by learning from examples generated in a rigid body physics simulation. Finally, at the grasp selection stage we now optimise across a space defined by this bundle of approach trajectories so as to maximise the chance of reaching a stable grasp. The method copes with partial and noisy shape information for the test objects. 

% SUBSECTION BETTER?

\subsection{Related Work}
Previous work in learning generalisable grasps falls broadly into two classes. One class of approaches utilises the shape of common object parts or their appearance to generalise grasps across object categories \cite{saxena2008b,detry2013a,herzog2014a, kroemer2012a}. This works well for low DoF hands. Another class of approaches captures the global properties of the hand shape either at the point of grasping, or during the approach \cite{ben2012generalization}. This global hand shape can additionally be associated with global object shape, allowing generalisation by warping grasps to match warps of global object shape \cite{hillenbrand2012transferring}. This second class works well for high DoF hands, but generalisation is more limited. We have previously achieved the advantages of both classes, generalising grasps across object categories with high DoF hands. In this paper we go beyond this, learning and generalising grasps for under-actuated hands.

Several hands with such behavior have been proposed in the literature with different implementations~\cite{Catalano2014Adaptive, Dollar2010Highly}, with a common goal: simplicity plus robustness. Their initial tests under human operation are promising, but autonomous grasping with underactuated hands faces challenges due to the almost non-observability of the finger deformation when the hand is constrained by the environment and/or a target object. Most of the existing planning algorithms for this type of hands boil down to generating good wrist poses and let the adaptive mechanism handle all variation and uncertainty while closing, such as~\cite{Eppner2015Planning}, where a sequence of wrist and object poses and the corresponding interaction wrenches are generated, which are expected to exploit environmental constraints. Another approach is that by~\cite{Bonilla2015Grasp}, where static wrist poses are sampled using different strategies around the object from where the fingers are closed using a rigid-body simulator, to finally select the areas of major success rate to generate new wrist poses.

While these approaches exploit, to some extent, the adaptive properties of the underactuated mechanism, they can be improved on. In this paper we show how we can, for the first time, learn grasps for underactuated hands that are then transferred to novel objects. This requires learning representations of the final grasp state that are amenable to transfer to new objects, grouping example grasps by the end grasp state, and learning and optimisation of reach-to-grasp trajectories.


%!TEX root = dexterous-grasping-sage-style.tex

\section{OVERVIEW OF APPROACH}
\label{sec:overview}
In our approach the main steps are as follows. A model of a training object is presented in a rigid body physics simulator. Then a number of example grasps are executed by a human, with the precise motions of hand and object during contact being determined by the simulation. Each example grasp continues until a final stable grasp state is reached. We call this the {\em equilibrium state}, consisting of the final hand shape, and the final set of contact relations between hand and object. For training and inference purposes each example grasp has two parts: an equilibrium state, and the reach to grasp trajectory leading to it.

We generate the example grasps in sets. Each set corresponds to a type of grasp, e.g. power or pinch. This means that the equilibrium states are similar within a set, but differ substantially between sets. 

Models are then learned for each grasp and for each set. Models are learned of the reach to grasp, the hand configuration in the equilibrium state, and the contact relations between hand and object in the equilibrium state. Given these models, when a new object is presented, a (partial) model of that object is obtained by sensing. This model is combined with the models learned from the training grasps.

Then many candidate equilibrium states, and associated candidate reach to grasp trajectories are generated by sampling. Finally they are optimised so as to maximise the likelihood of the grasp according to a product of experts.

\section{BASIC REPRESENTATIONS}
\label{sec:representations}
We now sketch the representations underpinning our approach. We define several models: an object model (partial and acquired from sensing); a model of the contact between a finger link and the object; a model of the whole hand configuration; and a model of the reach to grasp trajectory. First we describe the kernel density representation for all these models. Then we describe the surface features we use to encode some of these models. Then we follow with a description of each model type. Throughout we assume that the robot's hand comprises $N_L$ rigid \emph{links}: a palm, and several phalanges or links. We denote the set of links $L =\{L_i\}$. 

\subsection{Kernel Density Estimation}
\label{sec:kde}
$SO(3)$ denotes the group of rotations in three dimensions. A feature belongs to the space $SE(3) \times \mathbb R^2$, where $SE(3) = \mathbb R^3 \times SO(3)$ is the group of 3D \emph{poses}, and surface descriptors are composed of two real numbers. We extensively use probability density functions (PDFs) defined on $SE(3) \times \mathbb R^2$.  We represent these PDFs non-parametrically with a set of $K$ features (or particles) $x_j$
\begin{equation}
S = \left\lbrace x_j : x_j \in \mathbb R^3 \times SO(3) \times \mathbb R^2 \right\rbrace_{j \in [1,K]}.
\end{equation}
The probability density in a region is determined by the local density of the particles in that region. The underlying PDF is created through \emph{kernel density estimation} \cite{silverman1986a}, by assigning a kernel function $\mathcal{K}$ to each particle supporting the density, as
\begin{equation}\label{eq:d}
\pdf(x) \simeq \sum_{j=1}^K w_j \mathcal{K}(x| x_{j}, \sigma),
\end{equation}
where  $\sigma \in \mathbb R^3$ is the kernel bandwidth and $w_j \in \mathbb R^{+}$ is a weight associated to $x_j$ such that $\sum_j w_j = 1$. We use a kernel that factorises into three functions defined by the separation of $x$ into $p \in \mathbb R^3$ for position, a quaternion $q \in SO(3)$ for orientation, and $r \in \mathbb R^2$ for the surface descriptor. Furthermore, let us denote by $\mu$ another feature, and its separation into position, orientation and a surface descriptor. Finally, we denote by $\sigma$ a triplet of real numbers:
\begin{subequations}
\begin{align}
x &= (p, q, r),\\
\mu &= (\mu_p, \mu_q, \mu_r),\\
\sigma &= (\sigma_p, \sigma_q, \sigma_r).
\end{align}
\label{eq:feature}
\end{subequations}
We define our kernel as
\begin{equation}\label{eq:kernel_in_se3}
\mathcal{K}(x| \mu, \sigma) = \mathcal{N}_3(p| \mu_p, \sigma_p) \Theta(q| \mu_q, \sigma_q) \mathcal{N}_2(r| \mu_r, \sigma_r)
\end{equation}
where $\mu$ is the kernel mean point, $\sigma$ is the kernel bandwidth, $\mathcal{N}_n$ is an $n$-variate isotropic Gaussian kernel, and ${\Theta}$ corresponds to a pair of antipodal von Mises-Fisher distributions which form a Gaussian-like distribution on $SO(3)$ \cite{fisher1953a,sudderth2006a}. The value of ${\Theta}$ is given by
\begin{equation}
\Theta(q|\mu_q, \sigma_q) = C_4(\sigma_q) \frac {e^{\sigma_q \; \mu_q^T q} + e^{-\sigma_q \; \mu_q^T q}}2
\end{equation}
where $C_4(\sigma_q)$ is a normalising constant, and $\mu_q^T q$ denotes the quaternion dot product.

Using this representation, conditional and marginal probabilities can easily be computed from \eq\eqref{eq:d}. The marginal density $\pdf(r)$ is computed as
\begin{align}\label{eq:marginal}
\pdf(r) & \\
      = & \iint \sum_{j=1}^K w_j \mathcal{N}_3(p| p_i, \sigma_p) \Theta(q| q_i, \sigma_q) \mathcal{N}_2(r| r_i, \sigma_r) \textnormal{d}p\textnormal{d}q \\
      = &  \sum_{j=1}^K w_j \mathcal{N}_2(r| r_j, \sigma_r),
\end{align}
where $x_j = (p_j, q_j, r_j)$.
The  conditional density $\pdf(p,q|r)$ is given by
\begin{align}\label{eq:conditional}
\pdf(p,q|{r}) & = \frac{\pdf(p, q, {r})}{\pdf({r})} \\
                   & = \frac{\sum_{j=1}^K w_j \mathcal{N}_2({r}| r_j, \sigma_r) \mathcal{N}_3(p| p_j, \sigma_p) \Theta(q|q_j, \sigma_q)}{\sum_{j=1}^K w_j \mathcal{N}_2({r}| r_j, \sigma_r)}. 
\end{align}

\subsection{Surface Features}
\label{sec:surface_features}

All objects considered in the paper are represented by point clouds constructed from one or multiple shots taken by a depth camera. We directly augment these points with a frame of reference and a surface feature that is a local curvature descriptor. For compactness, we also denote the pose of a feature as $v$. As a result,
\begin{equation}
x = (v, r), \qquad v = (p, q).
\label{eq:surface.feature}
\end{equation}

The surface normal at $p$ is computed from the nearest neighbours of $p$ using a PCA-based method (e.g. \cite{kanatani2005statistical}). The surface descriptors are the local \emph{principal curvatures} \cite{spivak1979comprehensive}. Their directions are denoted $k_1, k_2 \in \mathbb R^3$, and the curvatures along $k_1$ and $k_2$ form a $2$-dimensional feature vector $r = (r_1, r_2) \in \mathbb R^2$. %The surface normals and curvatures are computed using the PCL library \citep{Rusu_ICRA2011_PCL}.
The surface normal and the principal directions define the orientation $q$ of a frame that is associated with the point $p$. 

\subsection{Object Model}
\label{sec:object_model}

Thus, given a point cloud, a set of $K_O$ features  $\lbrace (v_j, r_j) \rbrace$ can be computed. This set of features defines, in turn, a joint probability distribution, which we call the \emph{object model}:
\begin{equation}
\om(v, r) \equiv \pdf^\om(v, r) \simeq \sum_{j=1}^{K_O} w_j \mathcal{K}(v, r|{x_j}, \sigma_{x})
%RD: mu and sigma are not properly defined.
\label{eq:om}
\end{equation}
where $\om$ is short for $\pdf^\om$, $x_j = (v_j, r_j)$,  $\mathcal{K}$ is defined in \eq\eqref{eq:kernel_in_se3} with bandwidth $\sigma_{x} = (\sigma_{v}, \sigma_{r})$, and where all weights are equal, $w_j = 1/{K_O}$. In summary this object model $\om$ represents the object as a pdf over surface normals and curvatures.

\section{LEARNED MODELS}

We now describe the representations for each of the three models that are learned from a set of grasp examples. We start with the contact model, proceed with the equilibrium state hand configuration model, and finish with the reach to grasp model.

\subsection{Contact Model}\label{sec:contact.model}

A contact model $\cm_i$ encodes the joint probability distribution of surface curvatures and of the 3D pose of the $i$-th hand link in the equilibrium state. Let us consider the hand as having grasped some given training object. The contact model for link $\rl_i$ is denoted by
\begin{equation}\label{eq:M}
\cm_i(U, R) \equiv \pdf^\cm_i(U, R)
\end{equation}
where $\cm_i$ is short for $\pdf^\cm_i$, $R$ is the random variable modelling surface curvature, and $U$ models the pose of $\rl_i$ \emph{relative} to the frame of reference defined by the directions of principal curvature and the surface normal. In other words, denoting realisations of $R$ and $U$ by $r$ and $u$, $\cm_i(u, r)$ is proportional to the probability of finding $\rl_i$ at pose $u$ relative to the frame of a nearby object surface patch that exhibits principal curvatures equal to $r$.

Given a set of features $\lbrace x_j \rbrace_{j=1}^{K_O}$, with $x_j = (v_j, r_j)$ and $v_j = (p_j, q_j)$, a contact model $\cm_i$ is constructed from them.  Features close to the link surface are more important than those lying far from the surface. Features are thus weighted, to make their influence on $\cm_i$  decrease with their distance to the $i^\textnormal{th}$ link. %(\fig\ref{fig:representations.modeldist.cont}). 
We use a weighting function whose value decreases exponentially with the square distance to the link:
\begin{equation}
w_{ij} = \begin{cases}\exp(-\lambda ||p_j-a_{ij}||^2) \quad &\textnormal{ if } ||p_j-a_{ij}|| < \delta_i\\
0 \quad &\textnormal{ otherwise},\end{cases}
\label{eq:learning.modeldist.wgh}
\end{equation}
where $\lambda \in \mathbb R^{+}$, there is a cut-off distance $\delta_i$, and $a_{ij}$ is the point on the surface of $\rl_i$ that is closest to $p_j$. The intuitive motivation for this choice is that we require a weight function that falls off quickly so that the contact model will only take account of the local shape, while falling off smoothly.

Let us denote by $u_{ij} = (p_{ij}, q_{ij})$ the pose of $\rl_i$ relative to the pose $v_j$ of the $j^{\mathnormal{th}}$ surface feature. In other words, $u_{ij}$ is defined as
\begin{equation}
u_{ij} = v_j^{-1} \circ s_i,
\label{eq:local.pose}
\end{equation}
where $s_i$ denotes the pose of $\rl_i$, $\circ$ denotes the pose composition operator, and $v_j^{-1}$ is the inverse of $v_j$, with $v_j^{-1} = (-q_j^{-1}p_j, q_j^{-1})$ (see \fig\ref{fig:representations.model}). The contact model is estimated as
\begin{equation}
\cm_i(u,r) \simeq \frac 1Z \sum^{K_{M_i}}_{j=1} w_{ij}\mathcal{N}_3(p|{p_{ij}}, \sigma_{p}) \Theta(q|{q_{ij}}, \sigma_{q}) \mathcal{N}_2(r|{r_j}, \sigma_{r})
\label{eq:cm}
\end{equation}
where $Z$ is a normalising constant, $u = (p, q)$, and where $K_{M_i} \leq K_O$ is a number of features which are within cut-off distance $\delta_i$ to the surface of link $\rl_i$. If the number of features $K_{M_i}$ of contact model $\cm_i$ is not sufficiently large, contact model $\cm_i$ is not instantiated and is excluded from any further computation. Consequently, the overall number of contact models $N_M$ is usually smaller than the number of links $N_L$ of the robotic hand. We denote the set of contact models learned from a grasp example $g$ as $\mathcal{M}^g=\{\mathcal{M}^g_i\}$. 
%Sum~\eqref{eq:cm} involves only terms for which $x_j = ((p_j, q_j), r_j)$ belong to the neighbourhood of $\rf_i$, $\lbrace x_j: ||p_j-a_{ij}|| \leq \delta, \delta \in \mathbb R^{+} \rbrace$. If the neighbourhood of a particular link $i$ is empty, i.e. $K_{M_i} = 0$, the corresponding contact model is not instantiated and it is excluded from any further computation.
\begin{figure}[t]
\centerline{\includegraphics[width=5cm]{resources/model}}
\caption[Contact model]{Contact model. The figure shows the $i$-th link $\rl_i$ (solid block) and its pose $s_i$. The black dots are samples of the surface of an object. The distance $a_{ij}$ between a feature $v_j$ and the closest point on the link's surface is shown. The rounded rectangle illustrates the cut-off distance $\delta_i$. The poses $v_j$ and $s_i$ are expressed in the world frame $W$. The arrow $u_{ij}$ is the pose of $\rl_i$ relative to the frame for the surface feature $v_j$.}
\label{fig:representations.model}
\end{figure}

The parameters $\lambda$ and $\sigma_{p}$, $\sigma_{q}$, $\sigma_{r}$ were chosen empirically. The time complexity for learning each contact model from an example grasp is $\Omega(T K_O)$ where $T$ is the number of triangles in the tri-mesh describing the hand links, and $K_O$ is the number of points in the object model.

%\begin{figure}[t]
%\centering{
%\subfloat[]{\includegraphics[height=3.4cm]{resources/example1}}\quad
%\subfloat[]{\includegraphics[height=3.4cm]{resources/example2}}\quad
%\subfloat[]{\includegraphics[height=3.4cm]{resources/example5}}
%}
%\caption[Hand-object contact]{Example top grasp of a mug represented by a point cloud (panel a). The dotted regions are rays between features and the closest hand link surfaces (panel b). The black curves with frames at the fingertips represent the range of hand configurations in \eq\eqref{eq:hc} (panel c).}
%\label{fig:representations.modeldist.cont}
%\end{figure}

\subsection{Equilibrium State Hand Configuration Model}

The equilibrium state hand configuration model, denoted $h^e_c(j) \in \mathbb R^D$, for the grasp examples $j = 1 \ldots k$ within the set of $k$ training examples of a particular grasp type $g$. The purpose of this model is to restrict the grasp search space (during grasp transfer) to hand configurations that resemble those observed during training. We combine the configurations for the examples $j = 1 \ldots k$ to create a single mixture model density:
\begin{equation}
\hc_g(h^e_c) \equiv \sum_{j=1}^{k} \mathcal{N}_D(h^e_c|h^e_c(j), \sigma_{h^e_c}) 
\label{eq:hc}
\end{equation}
This expresses a density over hand configurations in the equilibrium state for a grasp type $g$.

\subsection{Reach to Grasp Model}

For a particular grasp type, in addition to modelling the equilibrium states of the hand, we must also model the trajectories taken to reach those equilibrium states. A single reach to grasp trajectory for an underactuated hand has three elements: the tool centre point (wrist) trajectory, the hand configuration trajectory, and the motor signal trajectory. We assume that a trajectory starts at time $t_0$ and ends in the equilibrium state at time $t_e$. We denote the wrist trajectory $h_w^{0:e}$, the hand configuration trajectory $h_c^{0:e}$, and the motor signal trajectory $h_m^{0:e}$ respectively. The motor signal can be a wide variety of signals in practice. Here we choose it to be the position of the single actuator. When the hand is not in contact with an object the motor signal and the wrist pose together determine the hand configuration. When in contact the actual hand configuration will differ. The reach to grasp model is simply the concatenation of each component $(h_w^{0:e} , h_c^{0:e} , h_m^{0:e})$. The set of reach to grasp trajectory models defining an attractor basin, leading towards the final hand configuration.

In the next section we explain how we gather the grasp examples that are used to learn these models. Then in Section~\ref{sec:infer} the inference method---by which the models are used to generate grasps for new objects---is described.


\section{DATA GENERATION}
% SAY HOW WE USE THE SIMULATOR HERE
% DESCRIBE BRIEFLY THE HAND

There are several ways to implement underactuation in a dexterous hand. In this paper we employ an approach based on adaptive synergy transmission, due to its simplicity and robust design, and its ability for complex interaction with the environment. The Pisa/IIT SoftHand~\cite{Catalano2014Adaptive} implements such a transmission mechanism. This hand has 19 degrees of freedom (DoF) distributed over four fingers and an opposable thumb, but only 1 degree of actuation (DoA). The synergy motion of the hand in free space has been derived from databases of human hand postures. The overall behaviour parameters are the matrices that correspond to the transmission ratio,~$R$, to the joint stiffness,~$K_q$. The actuation is done through a single tendon routed throuh all joints, making the fingers flex and abduct.

% THE LACK OF SENSORS AND FUTURE LARGE DATASETS REQUIRED JUSTIFY THE USE OF SIMULATED DATA

Moving such a hand to grasp an object results in a hard-to-predict contact and hand shapes due to the adaptivity. We thus generate a variety of grasp examples to cover a portion of the interaction space. However, recording many trajectories of all the finger elements that affect the grasp in the real world is non-trivial. For this reason, we generate the example interactions for training using a rigid-body physics simulator, where these problems are avoided. The main two simulation elements we have developed are the contact stability model and the hand behavior model. In the case of the Pisa/IIT softhand, the latter depends heavily on the former. We used the standard distribution of Gazebo and Open Dynamic Engine, both in widespread use. The adaptive synergy equations have been implemented as a plugin to these, and accompany the proper kinematic description of the Pisa/IIT SoftHand\footnote{The Pisa/IIT SoftHand ROS/Gazebo packages are available at \url{https://github.com/CentroEPiaggio/pisa-iit-soft-hand}}.

\begin{figure}
  \centering
  \begin{tabular}{ccc}

  \includegraphics[width=0.3\linewidth]{containerB_pinch_1.png} &

  \includegraphics[width=0.3\linewidth]{containerB_rim_1.png} &

  \includegraphics[width=0.3\linewidth]{pot_1.png} \\

  \includegraphics[width=0.3\linewidth]{containerB_pinch_2.png} &

  \includegraphics[width=0.3\linewidth]{containerB_rim_2.png} &

  \includegraphics[width=0.3\linewidth]{pot_2.png} \\

  \includegraphics[width=0.3\linewidth]{containerB_pinch_4.png} &

  \includegraphics[width=0.3\linewidth]{containerB_rim_4.png} &

  \includegraphics[width=0.3\linewidth]{pot_3.png} \\

  \end{tabular}

  \caption{Snapshots of the simulation of a pinch and rim grasp types for the colander (first and second columns), and handle grasp for the pot (last column).}
  \label{fig:simulations}
\end{figure}

%% HOW WE GENERATED THE DATA
%
At the current state, there are no generally accepted measures concerning whether a grasp by an underactuated hand  is good or not, hence the lack of robust grasp planners for them is not a surprise. Thus, generating a large dataset at this point is useless, and there are plans in the future to cover this area. As a result, we generated the examples by guiding manually the hand to a ``nice'' grasp as shown in Fig.~\ref{fig:simulations}. In this simpler scenario, we assume without loss of generality that the grasps are labelled by type. In our example dataset, we have three grasp types namely pinch, rim and by-the-handle. The main difference between pinch and rim is the fingers configuration w.r.t. the top border of objects. In the pinch grasp, the thumb goes inside whereas in the rim grasp, the fingers go inside. In the latter, the container can be filled with liquid while holding, for instance.
%
For each grasp example, the corresponding dataset comprises the set of trajectories as described in the previous section.

%For each grasp example, the corresponding dataset comprises the set of body trajectories, expicitly their position, $p_i(t) \in \mathbb{R}^3$, and orientation, $q_i(t) \in SO(3)$, over time for $i=1 , \ldots , m$ bodies (including hand links), and an associated set of hand configurations, $h_c(t) \in R^{n}$, with $n$ the hand degrees of freedom.
%%associated set of hand configurations, %$h_c(t) \in R^{n}$, with $n$ the hand degrees of freedom.

% FOR FUTURE WORK?

% Another challenge is to learn how any adaptive synergy transmission behave for any given kinematic design. 

%% BECAUSE I'M WORKING IN GENERATING DATA FOR ANOTHER COMPLIANT HAND BESIDES THE PISA SOFTHAND

\section{INFERENCE}
\label{sec:infer}
% NEED TO ADAPT FOR SOFT HAND (TRAJECTORY + CONTACT MODEL)
\section{Inference}
\label{sec:infer}

After acquiring the models from a set of training grasps, we present the robot with a test (query) object. The goal is to find a generalisation of the training grasp that is likely according to each model type. First of all, we obtain a point cloud for the test object, and thus an object model. We then combine every contact model with that object model, so as to obtain a set of {\em query densities}, one for each link with a contact model defined for the example grasp. The $i$-th query density $\qd_i$ is a density modelling where the $i$-th link can be placed, with respect to the surface of a new object. From the query densities, a candidate grasp is generated as follows. We randomly pick a link $i$. We randomly sample, from the corresponding query density $\qd_i$, a pose for link $i$. We sample, from the configuration model $\hc$, a final hand configuration that is compatible with the pose selected for link $i$, and then we compute from forward kinematics the poses of all the remaining hand links. Finally we sample at random, a trajectory from the set leading to the final grasp state. We refine this entire reach to grasp by performing a simulated annealing search in the space of hand poses and configurations, so as to locally maximise the grasp likelihood, Grasp likelihood is the product of the hand configuration density, the reach to grasp density, and the query densities for all the hand links. We repeat the entire process a number of times.

The optimisation procedure generates many possible grasps, each with its likelihood. The following subsections explain how to estimate query densities, and how grasp optimisation is carried out.

\subsection{Query Density Computation}

A query density \eqref{eq:qd} is a density for a  $K_{Q_i}$ kernels centred on the set of weighted finger link poses returned by \alg\ref{alg:mc}:
\begin{equation}
\qd_i(s) \simeq \sum^{K_{Q_i}}_{j=1} w_{ij} \mathcal{N}_3(p|{\hat{p}_{ij}}, \sigma_{p}) \Theta(q|{\hat{q}_{ij}}, \sigma_{q})%, \quad i = 1, ..., N_L
\label{eq:qd.approx}
\end{equation}
with $j$-th kernel centre $({\hat{p}_{ij}}, {\hat{q}_{ij}}) = \hat{s}_{ij}$, and where all weights were normalised $\sum_j w_{ij} = 1$. The number of kernels $K_{Q_i} = K_Q$ were chosen equal for all query densities and grasp types (unless otherwise stated). The non-Euclidean domain on which our density estimates are computed makes it difficult and computationally expensive to find optimal values for the bandwidth $\sigma_{p}$ and $\sigma_{q}$. Instead, we set the values of the bandwidths $\sigma_{p}$ and $\sigma_{q}$ using Silverman's popular rule of thumb \cite{silverman}. Silverman's rule does not give optimal bandwidth values, but it has strong empirical support and it yielded good results in our experiments. \fig\ref{fig:grasping.querydensity} depicts two example query densities created for two contact models of a handle grasp.

When a test object is presented a set of query densities $\mathcal{Q}^g$ is calculated for each training grasp $g$. The set $\mathcal{Q}^g =\{\qd_i^g\}$ has $N^g_Q=N^g_M$ members, one for each contact model $M_i^g$ in $\mathcal{M}^g$. The computation of each  query density has time complexity $\Omega( K_{M_i} K_Q)$ where $K_{M_i}$ is the number of kernels of the $i$-th contact model density~\eqref{eq:cm}, and $K_Q$ is the number of kernels of the corresponding query density. 

\section{RESULTS}
\label{sec:results}
% DESCRIBE EXPERIMENTS
The experiments were conducted as follows. Training consisted of nine example grasps, executed in simulation, with a human in control. These nine grasps were grouped into three grasp types (rim, pinch, and handle). The rim and pinch grasp types were trained on the colander object, and the handle grasp type was demonstrated on the saucepan.

\begin{figure}
 \includegraphics[width=0.9\columnwidth]{images/object_set}
 \caption{The two training objects are on the far left. The testing objects on the right. 12 from 15 test grasps on novel objects were successful.}
 \label{fig:test}
\end{figure}

 

%%%%%%%%%%%%%%%%%%%%%%%%%%%%%%%%%%%%%%%%%%%%%%%%%%%%%%%%%%%%%%%%%%%%%%%%%%%%%%%%
\bibliographystyle{IEEEtran}
%\bibliography{bibliography/carlos,bibliography/ref1,bibliography/ref2}
%% THERE ARE REPEATED ENTRIES IN REF1 AND REF2
\bibliography{bibliography/carlos,bibliography/ref1,bibliography/ref2}

\addtolength{\textheight}{-12cm}

%\section*{APPENDIX}
%
%
%\section*{ACKNOWLEDGMENT}


\end{document}
